\documentclass[a4paper]{article}
%\documentclass[a4paper]{report}
\usepackage{graphics}
\usepackage{breqn}
\setlength{\parindent}{0mm}
\setlength{\parskip}{1.8mm}
\newtheorem{definition}{Definition}
\newtheorem{theorem}{Theorem}
\newtheorem{proof}{Proof}
\title{On model reduction and equivalence}
%%\date{$28^{th}$ August, 2007}
\author{Daniel Tang}
%%\linespread{1.3}

\begin{document}
\maketitle
\section{Poisson models and Markov processes}

Let a Poisson model be a dynamic system on some discrete state space such that the probability that the process will transition from state $x$ to state $x'$ in time $dt$ is given by $\rho('x|x)dt$. If we let $\rho_{ij}$ be a transition matrix such that
\[
\rho_{ij} = 
\begin{cases}
\rho(i|j) & \text{if } i \ne j\\
\sum_{k \ne j} -\rho(k|j) & \text{if } i = j
\end{cases}
\]
and let $X$ be a vector representing a probability distribution over the state space, then we can define the Poisson model as a continuous time dynamic system
\[
\frac{dX}{dt} = \rho X
\]
so that
\[
X(t) = e^{\rho t}X(0) = e^{(\mu - I)rt}X(0) = \sum_{k=0}^\infty \frac{ (rt)^k e^{-rt}}{k!}\mu^kX(0)
\]
where $r = \max_j \sum_{k \ne j} \rho(k|j)$ is a scalar and $\mu = \frac{\rho}{r} + I$. It can be seen that each entry of $\mu$ is non-negative and the sum of entries in each column is 1, so $\mu^k X(0)$ is the state of a discrete time Markov process at time $k$.

Since $\mu rt$ commutes with $Irt$ we can separate and expand to give
\[
X(t) = e^{-rt}e^{\mu rt}X(0) = \sum_{k=0}^\infty \frac{ (rt)^k e^{-rt}}{k!}\mu^kX(0)
\]
which we recognize as a sum of powers of $\mu$ weighted with a Poisson distribution. So, each continuous-time Poisson model $\rho$ has an associated discrete-time Markov process $\mu = \frac{\rho}{r} + I$. The two are related by the fact that, for a given start state distribution, the state of the Poisson model at time $t$ is the weighted sum of states in a trajectory of the Markov process, where the weights are given by a Poisson distribution with rate $rt$.

This means, among other things, that as $t \to \infty$ the state distribution of a Poisson model tends to a uniform distribution over the attractor of its associated Markov model. So every Poisson model tends to a steady state distribution (i.e. a single point in distribution space).

\section{Social norms and social agents}

Let a \textit{social role} be a tuple $\left<S, \sigma_0, A, Q\right>$ where $S$ is a domain of social states of the actor of which $\sigma_0$ is the default state given at the beginning of an episode, $A$ is a domain of social actions that can be performed by the actor and $Q$ is a social-quality function in $S \times A \to \mathcal{R}$. 

Let a \textit{social norm} be defined as a tuple $\left<r_1, r_2, \tau \right>$ where $r_1$ and $r_2$ are social roles and $\tau$ is a state transition function in $S_1 \times A \times S_2 \to S_1 \times S_2$ saying that if two agents playing roles $r_1$ and $r_2$ are in states $s_1$ and $s_2$ respectively and agent 1 performs action $a$ on agent 2 then, after the action the agents have state $s'_1$ and $s'_2$ respectively.

Let a \textit{social connection} be a tuple $\left<N, s_1, s_2 \right>$ where $N$ is a social norm, $s_1$ is the social state of the first agent in $N$ and $s_2$ is the social state of the second.

Let a \textit{social network} be a directed graph, where each edge is associated with a social connection. So, each agent in a social network can be thought of as being in a number of social states (one for each edge) and each social state comes with a constraint on behaviour in the form of the social Q-functions.

Let a \textit{social agent} be an agent who models their world as a social network in which they are one of the agents. The agent uses their own social state as a guide on how to behave and other's social states to help them predict other's behaviour when deciding how to behave. 

Agents in a social network need not, and often do not, physically exist. God, the government and Microsoft do not exist but many people model them as agents to which they have social connections, and these connections can be instrumental in the coordination of behaviour between physically existant agents.

In many situations, we're only interested in the social actions that each agent performs, so we can model a set of agents as interacting via social actions. In reality, these social actions are grounded in physical actions and if we're interested in the details of this, then each social action must come with an extension which is a set of sequences of physical actions that qualify as this social action. The agent must then choose a physical action in consideration of the social interpretations of that action. In general, the physical substrate of the agent's social world adds additional constraints to the sequence of social actions it may perform.

Modelling at the physical level also allows agents to perform actions that have multiple interpretations, so a single physical action or sequence of actions can qualify as mamy social actions. It is also possible for a physical action to be unclassifiable as a social action yet state changing in the social norm (e.g. I sit down at a restaurant, the waiter brings me the menu, i set fire to it). In this case the norm is considered null and new norms need to be negotiated.

\section{How do agents learn social norms?}

Language, telling stories. By accidentally breaking social norms.

Can we learn social norms without language?

\section{how do agents negotiate roles to form new social connections?}
Context. Verbally.

\end{document}