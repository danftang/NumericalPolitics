\documentclass[a4paper]{article}
%\documentclass[a4paper]{report}
\usepackage{graphics}
\usepackage{breqn}
\setlength{\parindent}{0mm}
\setlength{\parskip}{1.8mm}
\newtheorem{definition}{Definition}
\newtheorem{theorem}{Theorem}
\newtheorem{proof}{Proof}
\title{On model reduction and equivalence}
%%\date{$28^{th}$ August, 2007}
\author{Daniel Tang}
%%\linespread{1.3}

\begin{document}
\maketitle
\section{Poisson models and Markov processes}

Let a Poisson model be a dynamic system on some discrete state space such that the probability that the process will transition from state $x$ to state $x'$ in time $dt$ is given by $\rho('x|x)dt$. If we let $\rho_{ij}$ be a transition matrix such that
\[
\rho_{ij} = 
\begin{cases}
\rho(i|j) & \text{if } i \ne j\\
\sum_{k \ne j} -\rho(k|j) & \text{if } i = j
\end{cases}
\]
and let $X$ be a vector representing a probability distribution over the state space, then we can define the Poisson model as a continuous time dynamic system
\[
\frac{dX}{dt} = \rho X
\]
so that
\[
X(t) = e^{\rho t}X(0) = e^{(\mu - I)rt}X(0) = \sum_{k=0}^\infty \frac{ (rt)^k e^{-rt}}{k!}\mu^kX(0)
\]
where $r = \max_j \sum_{k \ne j} \rho(k|j)$ is a scalar and $\mu = \frac{\rho}{r} + I$. It can be seen that each entry of $\mu$ is non-negative and the sum of entries in each column is 1, so $\mu^k X(0)$ is the state of a discrete time Markov process at time $k$.

Since $\mu rt$ commutes with $Irt$ we can separate and expand to give
\[
X(t) = e^{-rt}e^{\mu rt}X(0) = \sum_{k=0}^\infty \frac{ (rt)^k e^{-rt}}{k!}\mu^kX(0)
\]
which we recognize as a sum of powers of $\mu$ weighted with a Poisson distribution. So, each continuous-time Poisson model $\rho$ has an associated discrete-time Markov process $\mu = \frac{\rho}{r} + I$. The two are related by the fact that, for a given start state distribution, the state of the Poisson model at time $t$ is the weighted sum of states in a trajectory of the Markov process, where the weights are given by a Poisson distribution with rate $rt$.

This means, among other things, that as $t \to \infty$ the state distribution of a Poisson model tends to a uniform distribution over the attractor of its associated Markov model. So every Poisson model tends to a steady state distribution (i.e. a single point in distribution space).

\section{Social frames}

Suppose an agent knows it is in a Poisson model and has a set of observations of actions and interactions. The agent would like to set action rates given observations under the assumption that it is in a Poisson model (plus, likely, some additional assumptions). Algorithms for doing this can be rated against eachother by comparing their rewards under well defined conditions.

Our hypothesis is that agents do this using social norms which consist of multiple, incommensurable ``social frames''. In each social frame the agent plays a different role in relation to a set of other agents. At any point in time an agent's social state is defined by the set of frames that agent is currently active in. When multiple agents share a frame and play their role the agents can exhibit a sequence of coordinated behaviour which we call a ``story''.

\subsection{What is a social frame?}

A social frame has a number of roles. When active, each role is assigned to an agent (or object, or may be an imaginary entity), and we say that the agent becomes a ``player'' in the frame (perhaps we also have ``out-group'' role which can be played by any agent that isn't playing any other role in the abstraction). Each role comes with a state space and a set of social actions that may be performed on the role player. However, a social action is distinct from a physical action, so each social action comes with
\begin{enumerate}
	\item an abstraction function that allows agents to recognize sets of physical actions as social actions within the abstraction
	\item a concretization function that allows agents to perform social actions	
\end{enumerate}

Each role also has a model defining state change as a result of social action, and a function from state to set of socially acceptable social actions (and perhaps explicitly forbidden social actions?). If an agent performs a physical action that can't be classified as a social action within the frame then the whole story breaks down.

So, a social frame is a Poisson ABM with a small number of agents with small state spaces, social actions and SocialQ-value state-action functions, and abstraction/concretisation functions between social action and physical action [is this necessary, or does each physical action have a (possibly null) meaning within the abstraction? Abstracted actions add heirarchical structure to actions, thereby making complex action easier (for example, me giving you a file which is currently on my USB stick may invilve a complex sequence of actions that may involve dealing with contingencies and entering into sub-interactions with you)]. If the agents in an abstraction have reward functions then the abstraction can be compared to the real reward functions of the players. If agents can berate/praise each other, and do it according to the social frame, and care about being berated/praised, the social frame becomes self-fulfilling.

\subsection{Fiat belief}

So, we can distinguish between physical and social belief, and physical and social state. Physical belief is a belief that the world is in some abstract state, and as such can be refuted by evidence. Social belief, on the other hand, concerns the state of an agent within a social frame and is neither true nor false but is an expression that one agent considers another agent as a player in a social frame. For example, if I say that you owe me £100, then I'm not saying something about your physical state in the way that I would be if I said that you had cancer. Instead, I am expressing an attitude towards you: I consider you to be a player in the lend/return social frame and you are in the state of oweing me £100. An agent can and will have attitudes towards herself. Alice can, of course, believe that she owes Bob £100.

So, the state of a social situation can be defined as the social attitudes each agent has towards others. The attitudes Alice has towards Bob strongly constrains the way Alice behaves towards him.

So, my social attitude towards you means that I will model your behaviour and interpret your actions within the frame of that attitude. A social attitude towards self gives us a model of how to behave. This is the true currency of social interaction. [Can all social situations be reduced to binary attitudes? Do we tend to reduce attitudes to social attributes (i.e. that the attribute inheres in the agent, rather than is an attitude towards an agent)?]

This explains how we can deal with imaginary entities (like God, Microsoft or the government). If a number of agents have attitudes towards an imaginary entity, and social actions on that entity are abstracted away from any physical actions on the entity, then there is no need for the entity to have any physical existance, yet the attitudes towards it can be instrumental in collective behaviour among real agents.


\section{How do agents learn social frames?}

Language, telling stories. By accidentally breaking social norms.

Can we learn social frames without language?

[how do agents negotiate roles to initiate a story?]
Context. Verbally.

\end{document}