\documentclass[a4paper]{article}
%\documentclass[a4paper]{report}
\usepackage{graphics}
\usepackage{breqn}
%%\usepackage[english,greek]{babel}
\setlength{\parindent}{0mm}
\setlength{\parskip}{1.8mm}
\newtheorem{definition}{Definition}
\newtheorem{theorem}{Theorem}
\newtheorem{proof}{Proof}
\title{Computation as occupation numbers}
%%\date{$28^{th}$ August, 2007}
\author{Daniel Tang}
%%\linespread{1.3}

\begin{document}
%%\selectlanguage{english}
\maketitle
%%\tableofcontents

Let $V$ be an $A$-dimensional vector of integers. Each element index represents a possible state of an agent and the integer value at that index represents the number of agents in that state. We'll call this an occupation vector. We define a transition rate, $\rho$, so that $\rho(V'|V)dt$ is the probability that a vector $V$ will transition to state $V'$ in time $dt$, so we have a Markov model with Poisson dynamics over the domain of occupation vectors.

We assume that each transition in the Markov model is either a single agent responding to its own state, irrespective of the other agents around it, or is an interaction between two agents. So, the behaviour all agents can defined by two agent-based action rates:
\begin{enumerate}
	\item $\rho_\psi(\Delta V)$ which is the rate at which an agent in state $\psi$ will perform an action that results in a perturbation in occupation vector of $\Delta V$ (i.e. $V' = V + \Delta V$) and
	\item $\rho_{\psi\phi}(\Delta V)$ which is the rate at which an agent in state $\psi$ will interact with an agent in state $\phi$ to produce a perturbation of $\Delta V$. 
\end{enumerate}

So, the total rate of transition 
\begin{equation}
\rho(V + \Delta V|V) = \sum_{\psi}  \rho_\psi(\Delta V) V_\psi + \sum_{\psi,\phi}  \rho_{\psi\phi}(\Delta V) \gamma(V_\psi,V_\phi)
\label{rateEq}
\end{equation}
where $\gamma(n,m)$ gives the ``aggregation rate'', for now we consider only $\gamma(n,m) = nm$ or $\gamma(n,m) = n(m>0)$.

Given this, we can consider probability distributions over the domain of occupation vectors. The transition rates define a rate of change of a probability distribution,
\begin{equation}
\frac{d\Phi(V)}{dt} = \sum_{V'} \rho(V|V')\Phi(V') - \rho(V'|V)\Phi(V)
\label{changeEq}
\end{equation}
In this way we translate the agent based model into a dynamic system whose state space is the probability distributions over the domain of occupation vectors. [is it true that all attractors are point attractors in this space? Does it matter?]

Substituting \ref{rateEq} into \ref{changeEq} and assuming $\gamma(n,m) = nm$
\begin{dmath}
\frac{d\Phi(V)}{dt} = 
\sum_{V'}\left(
\sum_{\psi}  \rho_\psi(V-V') V'_\psi
 + \sum_{\psi,\phi}  \rho_{\psi\phi}(V'-V) V'_{\psi}V'_{\phi}
 \right)\Phi(V') 
-
\left(
\sum_{\Delta V,\psi}  \rho_\psi(\Delta V) V_\psi 
+ \sum_{\Delta V,\psi,\phi}  \rho_{\psi\phi}(\Delta V) V_\psi V_\phi
\right)
\Phi(V)
\label{expandedchangeEq}
\end{dmath}

The occupation vectors can be thought of as points in an $A$-dimensional space, and $\Phi$ can be thought of as a probability field over an $A$-dimensional grid on the non-negtive integer coordinate points. With this in mind, we can simplify equation \ref{expandedchangeEq} by expressing the terms as convolutions. Let the ``transition kernel'' be
\[
\tau_\psi(\Delta V) = 
\begin{cases}
	\rho_\psi(\Delta V) & \text{if } \Delta V \ne 0\\
	\sum_{\Delta V'} -\rho_\psi(\Delta V') & \text{if } \Delta V = 0
\end{cases}
\]
and
\[
\tau_{\psi\phi}(\Delta V) = 
\begin{cases}
	\rho_{\psi\phi}(\Delta V) & \text{if } \Delta V \ne 0\\
	\sum_{\Delta V'} -\rho_{\psi\phi}(\Delta V') & \text{if } \Delta V = 0
\end{cases}
\]
and letting $n_\psi(V) = V_\psi$ so that the sums over $V'$ and $\Delta V$ become convolutions
\begin{equation}
\frac{d\Phi}{dt} = 
\sum_{\psi}\tau_\psi \ast n_\psi\Phi
+ \sum_{\psi,\phi}  \tau_{\psi\phi} \ast n_{\phi}n_{\psi}\Phi
\end{equation}

We can calculate higher order rates of change, since $\Phi$ is the only term with time dependency we have 
\begin{equation}
	\frac{d^{n+1}\Phi}{dt^{n+1}} = 
	\sum_{\psi}\tau_\psi \ast n_\psi \frac{d^n\Phi}{dt^n}
	+ \sum_{\psi,\phi}  \tau_{\psi\phi} \ast n_{\phi}n_{\psi} \frac{d^n\Phi}{dt^n}
\end{equation}


Now assume we have occupation matrices. The row gives the "interaction state" and the column gives the "agent type". Only agents in the same interaction state can interact.





%%\appendix

\end{document}
